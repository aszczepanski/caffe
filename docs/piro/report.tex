\documentclass[a4paper,11pt]{article}
\usepackage{polski}
\usepackage[utf8]{inputenc}
\usepackage[OT4]{fontenc}
\usepackage{latexsym}
\usepackage{fullpage}
\usepackage{color}
\usepackage{url}

\title{Music OCR}
\author{
	Adam Szczepański
	\and
	Piotr Żurkowski
}

\begin{document}

\maketitle

\section{Zasada działania}

\section{Wspierane znaki}

\section{Opis algorytmu}

\subsection{Wykrywanie linii i nut}
Pierwszym krokiem algorytmu jest znalezienie i usunięcie pięciolini z obrazu.
Po wstępnym thresholdowaniu i zmiany obrazu na binarny, wywoływana jest erozja i dylatacja, z kernelem w postaci macierzy 1 x width/30.
w ten sposób na obrazie zostają tylko poziome linie.
Podobna operacja wykonywana jest też wertykalnie, aby wspomóc proces ekstrakcji znaków opisany dalej.

Wyodrębniane linie, po dodatkowej dylatacji pionowej, odejmowane są od orginalnego obrazu.
W celu poprawy wyniku dodaje się wyodrębnione wcześniej elementy wertykalne.
Następnie obraz jest ponownie progowany, prowadząc do obrazu zawierającego same nuty.

W międzyczasie, wyekstraktowane linie przetwarzane są w celu znalezienia głównych pięciu, wykorzystywanych do szacowania wysokości.
Na obrazie sumowane są pixele w każdym wierszu, i progowane 30\% maksymalnej wartości.
W ten sposób otrzymywani są kandydaci na pięciolinię. następnie bliscy kandydaci łączeni są w jednych, po czym wybierane jest 5 najwyższych wartości.
Pozycje linii są zapisane, wraz ze średnią odległością między liniami.

\subsection{Segmentacja nut}
Segmentacja nut przeprowadzana jest na wyekstraktowanych symbolach z poprzedniego kroku.
Dla każdej kolumny sprawdzana jest zawartość czarnych pixeli i grupowane są kolejne kolumny które takie pixele zawierają.
Następnie łączone są grupy odległe o mniej niż połowa szerokości linii - w ten sposób unikając błędów wynikających z niedokładnych skanów lub niedokładnie narysowanych nut.
Jako wynik otrzymujemy listę grup w postaci zakresów kolumn, w których mogą znajdować się symbole.

Jako kolejny krok, wykonywane jest sprawdzenie, czy grupa zawiera pojedynczy symbol, czy połączoną grupę nut.
Przy obserwacji, że połączone nuty zawsze są pełne w środku, można podjąć się szukania główek nut w grupach jako wyznacznik czy grupa jest pojedynczym symbolem czy grupą nut.

Wyszukiwanie główek nut, odbywa się poprzez znalezienie w każdej kolumnie najdłuższego ciągu czarnych pixeli.
Kandydatami są ciąci o długości większej niż połowa wysokości linii, i mniejszej niż półtora wysokości linii.
kandydaci grupowani są w grupy, jeśli są odległe od siebie nie więcej niż 2 pixele (dopuszczenie szumów).
Kandydatami pozostają grupy, które są szerokie na przynajmniej połowę szerokości linii.
W ten sposób otrzymujemy listę możliwych główek nut.  Jeśli lista zawiera więcej niż jedną główkę nuty, jest traktowana jako połączone nuty.
Jeśli nie, traktowana jest jako pojedynczy symbol. Pojedyncze symbole przechodzą do fazy rozpoznawania i wykrywania wysokości.
Połączone traktowane są jako ósemki i przechodzą bezpośrednio do fazy wykrywania wysokości.

\subsection{Wykrywanie wysokości}
Dla wyciętego już obrazu nuty, wyszukuje się prawdopodobnej pozycji na pięciolinii.
Na początku przetwarzania, liczy się przedziały przynależności nut do wysokości.
Dla wcześniej wyznaczonych pozycjii pięciu linii, zapisywane są przedziały przynależności - dla położenia na linii -
$<PozycjaLinii - 0.25 * SzerokośćLinii, PozycjaLinii + 0.25 * SzerokośćLinii>$ - dla położenia pomiędzy liniami -
$<PozycjaDolnejLinii + 0.25 * SzerokośćLinii, PozycjaGórnejLinii - 0.25 * SzerokośćLinii>$.
Analogicznie szacowane są pozycje nut poniżej i powyżej pięciolinii.

Następnie dla nuty, liczone są pixele w każdym rzędzie, thresholdowane 30\% maksymalnej wartości, i grupowane w spójne przedziały.
Znajdowany jest najdłuższy przedział i jest traktowany jako główka nuty
(Wybrano najdłuższy przedział a nie najgęstrzy segment, ponieważ półnuty są puste w środku i mogą być łatwo przewyższone szumem z innych miejsc).
Dla znalezionego już przedziału, liczy się środek ciężkości jako wyznacznik położenia nuty.

Mając przewidywane położenie nuty, sprawdza się przynależność do przedziałów i przypisuję wysokość nucie.

\section{Podsumowanie}

Dużo informacji w~\cite{10strategies}.

\bibliography{report}
\bibliographystyle{plain}

\end{document}
